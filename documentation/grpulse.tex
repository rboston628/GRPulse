\chapter{Using the command line program \label{chapter:grpulse}}

\section{Installation and Set Up}

This program was developed in macOS and originally compiled with \tcode{gcc}.  If your \tcode{gcc} is up-to-date, then the included makefile should prepare everything for you.  Linux users should be able to make this work without too much intervention.  Windows users should consider installing a real operating system.

To compile, first \tcode{cd} to the \tcode{lib} directory and type
\begin{verbatim}
	make -f makelib
\end{verbatim}
This will create the library.  Next \tcode{cd} back to main directory and type \verb+make+.  That will create the entire program.

If you have persistent problems, check that you have the latest version of \tcode{gcc} installed, capable of C++11.  If so but there are still problems, please \href{mailto:\myemail}{email me.}

\section{Creating an input file}
The program includes a UI meant to be run from the command line, called \tfile{GRPulse}.  User specification for the stellar background and the modes should be placed in the input file.  The input file should be structured as follows:
\begin{itemize}
	\item Line 1: the word \tcode{Name: } followed by a keyword to be used when naming files related to this calculation.
	\item Line 2: the word \tcode{Model: } followed by the following specifications, separated by a space:
		\begin{itemize}
			\item a keyword for the \emph{regime} of physics:
				\begin{itemize}
					\item \tcode{newtonian} -- Newtonian physics
					\item \tcode{1pn} -- the first post-Newtonian approximation
					\item \tcode{gr} -- Einstein's general relativity
				\end{itemize}
			\item a keyword for the stellar background model:
				\begin{itemize}
					\item \tcode{polytrope} -- a polytrope \textcolor{blue}{(works in all regimes)}
					\item \tcode{MESA} -- a wrapper for a \tfile{MESA} model \textcolor{red}{(only in \tcode{newtonian})}
					\item \tcode{CHWD} -- a Chandrasekhar WD \textcolor{red}{(only in \tcode{newtonian}, \tcode{1pn})}
				\end{itemize}
			\item model-specific parameters.  See Sec \ref{sec:grpulse:params} below.
		\end{itemize}
	\item Line 3: in {\bf some models}, the word \tcode{Params: } followed by two of the following:
		\begin{itemize}
			\item the word \tcode{mass} and a real number indicating the total mass of the star
			\item the word \tcode{radius} and a real number indicating the total radius of the star
			\item the word \tcode{logg} and a real number indicating the $\log_{10} g$ surface gravity of star
			\item the word \tcode{zsurf} and a real number indicating the surface redshift
		\end{itemize}
	\item Line 4: in {\bf all models}, the word \tcode{Units: } followed by a keyword specifying:
		\begin{itemize}
			\item \tcode{CGS}, currently the main one working
			\item \tcode{geo}, for using geometric units as in GR
			\item \tcode{SI}, for using SI units
			\item \tcode{astro}, for using ``astronomical units'', where mass is in $M_\odot$, distance is in km, and time in seconds
		\end{itemize} Currently, only \tcode{CGS} and \tcode{astro} are confirmed to behave properly.
	\item Line 5: this must be a blank line.
	\item Line 6: in {\bf all models} the word \tcode{Frequencies: } followed by the mode type to use and the adiabatic index to use.  
		Mode type options are:
		\begin{itemize}
			\item \tcode{cowling} \textcolor{red}{(only in \tcode{newtonian} or \tcode{gr})}
			\item \tcode{nonradial}
		\end{itemize}
		The adiabatic index is a number.  The special cases $\Gamma_1=5/3$ and $\Gamma_1=4/3$ can be specified by writing the fractions \tcode{5/3} and \tcode{4/3}.  Other indices must be specified in decimal format.  To use the adiabatic index $\Gamma_1 = \left(\partial\log P/\partial\log\rho\right)_{ad}$ calculated from the stellar background then use \tcode{0}.
	\item Line 7: Beginning here and onward, write the mode numbers for each mode you wish to calculate on a separate line.  These are specified as $\ell, k$, where $k$ counts the nodes and $\ell$ the angular momentum.  Negative $k$ specify g-modes, positive specify p-modes.  Note that there is no $1,0$ mode, so specifying one can cause the program to hang. 
\end{itemize}
The file should be saved in the same file as \tfile{GRPulse}.  Assuming \tfile{GRPulse} has been properly compiled and the file called \tfile{myinput.txt}, then you can run with
\begin{verbatim}
	./GRPulse myinput.txt
\end{verbatim}

\subsection{Comments in Input Files}
Comments may be placed in an input file, subject to the following restrictions:
\begin{itemize}
	\item Comments must begin with \verb+#+.
	\item Comments may only be placed {\bf at the top of the file}.
	\item Blank lines may be used within the comment section, but not after. 
\end{itemize}
You can see this examples of comment use within the sample files.

\section{Parameters for stellar models \label{sec:grpulse:params}}
For the following model, on the same line as the model keyword, give the following parameters in order, separated by a space.
\begin{itemize}
	\item \tcode{polytrope} 
		\begin{enumerate}
			\item real number for polytrope index $n$
			\item integer for number of grid points
		\end{enumerate} 
		The next line should be \tcode{Params:} as above.
	\item \tcode{CHWD} 
		\begin{enumerate}
			\item real number for initial value $y_o$ (see \hyperlink{class:CHWD}{\tclass{ChandrasekharWD}})
			\item integer for number of grid points
		\end{enumerate}
		Do {\bf not} include \tcode{Params:} line.
	\item \tcode{MESA}
		\begin{enumerate}
			\item string with name of dat file (omitting the ``\tfile{.dat}'')
			\item integer for number of grid points to use -- this number is a \emph{suggestion}
		\end{enumerate}
		All \tcode{MESA} data files must be moved to the \tfile{GRPulse} directory.  
		Do {\bf not} include the \tcode{Params:} line.
\end{itemize}

\section{The output}
When the calculation begins, \tfile{GRPulse} will create a directory specific to this calculation based on the given name, located in the \tfile{output} directory.  The first output it makes is an echo of the input file, for re-running the specific calculation.  Assuming your calculation was given the name \tcode{myname}, this file will be called \tfile{myname\_in.txt}.  The main output file will be called \tfile{myname.txt}.  This contains data about the star, and summary results for all modes calculated.

The program will also create subdirectories called \tfile{modes} and \tfile{star}.  These contain graphs and data files relevant either to the stellar background or to the modes.  The \tfile{modes} directory can take up a lot of memory, so if you do not think you will need graphs of every mode, you may want to delete this directory's contents.

\section{Sample Input}
There are several sample inputs included in the distribution.  They perform the following:
\begin{description}
	\item[sampleinput1.txt] Will create results for an $n=0$ polytrope to compare against Pekeris formula.
	\item[sampleinput2.txt] Will create the post-newtonian results for $n=1$ in \mypaper
	\item[sampleinput3.txt] Will create the Newtonian results for $n=2$ in \mypaper
	\item[sampleinput4.txt] Will create a table which can be compared to results in Christensen-Dalsgaard \& Mullan (1994) paper for polytrope eigenmodes.
	\item[sampleinput5.txt] Shows how to use \tfile{MESA} data files.
\end{description}
